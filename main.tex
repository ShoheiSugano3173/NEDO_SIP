%%%%%%%%%%%%%%%%%%%%%%%%%%%%%%%%%%%%%%%%%%%%%%%%%%%
% 絶対に読み込まなければならないスタイル
%%%%%%%%%%%%%%%%%%%%%%%%%%%%%%%%%%%%%%%%%%%%%%%%%%%
% 標準のスタイルファイル jbook
% 11ポイント.奇数,偶数ページ同じデザイン.奇偶数どちらからでも章が始まる.
\documentclass[11pt,oneside,openany]{jbook} 
% 図を挿入できるように.dviout拡張を使う(Windows)
%\usepackage[dviout]{graphicx} 
\usepackage[dvipdfm]{graphicx} 
% AMS TeXの記号を利用する(数式記号拡張)
\usepackage{amsmath,amsthm,amssymb} 
% キャプションが長い場合に,折り返しをぶら下げにする
\usepackage[hang]{caption} 
% 章などの最初の段落の字下げを行う.
\usepackage{indentfirst} 
% 脚注をまとめる(E-Sysスタイルファイルで後注に変更)
\usepackage{endnotes} 
% ヘッダーとフッターの制御
\usepackage{fancyhdr}
%
%%%%%%%%%%%%%%%%%%%%%%%%%%%%%%%%%%%%%%%%%%%%%%%%%%%
% お勧めのスタイル.利用しなければつけなくて良い
%  各自がスタイルファイルを読み込ませたいときは
%  この部分に追加すること
%%%%%%%%%%%%%%%%%%%%%%%%%%%%%%%%%%%%%%%%%%%%%%%%%%%
% pdfファイルを作成する際に,リンクとしおりを作成.out2uniの場合
% 最近は dvipdfmxを使うので,この機能はOFFの方が良い
%\usepackage[dvipdfm, bookmarks=true,bookmarksnumbered=true,bookmarkstype=toc]{hyperref} 
% pdfファイルを作成するための拡張設定(しおりなどの作成)
\AtBeginDvi{\special{pdf:tounicode 90ms-RKSJ-UCS2}}
% 色を使う.pdfを作るときにも関係
\usepackage[dvipdfm]{color} 
% pdfファイルを作成するための設定
\usepackage[%
dvipdfm,%
pdfstartview={FitH -32768},%    描画領域の幅に合わせる[default は全体表示]
bookmarks=true,%                しおり付き[default は false]
bookmarksnumbered=true,%        章や節の番号をふる[default は false]
bookmarkstype=toc,%             目次情報のファイル[拡張子 toc]を参照
colorlinks=false,%              ハイパーリンクを色文字ではなく色枠に[default]
linkbordercolor={0 1 1},%       link の枠の色 aqua [default は red {1 0 0}]
citebordercolor={0 1 0},%       cite の枠の色 lime [default]
urlbordercolor={0 0 1},%        url の枠の色 blue [default]
%pdftitle={タイトル},%           [default は空白]
%pdfsubject={副タイトル},%       [default は空白]
%pdfauthor={氏名},%              [default は空白]
%pdfkeywords={キーワード}%       [default は空白]
]{hyperref}
% URLの記述が容易となる
\usepackage{url} 
% screen 環境が使えるように
\usepackage{ascmac} 
% box環境の拡張
\usepackage{eclbkbox} 
% 図に(a)(b)(c)と記号を振れるパッケージ
\usepackage{subcaption}
% コメントアウトを \begin{comment}コメント\end{comment}で実現
\usepackage{comment} 
% レイアウト確認の際に使う
%\usepackage{layout} 
% Verbatimの拡張
\usepackage{fancyvrb} 
% PSfragを利用する
%\usepackage[dviout]{psfrag} 
%\usepackage{psfrag} 
% EPS の図にラベルを重ねる
%\usepackage{labelfig}
% 表内にカラーを用いる
\usepackage[dviout]{colortbl}
% フォント
\usepackage{jtygm}
% 図の強制挿入
\usepackage{here}
%
%%%%%%%%%%%%%%%%%%%%%%%%%%%%%%%%%%%%%%%%%%%%%%%%%%%
% 絶対に読み込まなければならないスタイル
%  宮治が作成したり変更,拡張したもの
%%%%%%%%%%%%%%%%%%%%%%%%%%%%%%%%%%%%%%%%%%%%%%%%%%%
% nomenclスタイルの拡張,数式の記号一覧を作成するために使用
% nomenclスタイルを宮治修正
\usepackage[refeq,refpage,esys]{esysnomencl}
% E-Sys用スタイルファイル 宮治作成
\usepackage{esysthesis}
% 図・表関係のrefコマンド
\usepackage{sugarstyle}
%
%%%%%%%%%%%%%%%%%%%%%%%%%%%%%%%%%%%%%%%%%%%%%%%%%%%
% ユーザー設定領域
%  この部分を,自分(達)の情報に修正してください.
%%%%%%%%%%%%%%%%%%%%%%%%%%%%%%%%%%%%%%%%%%%%%%%%%%%
%
\jtitle{時間的非同期性を考慮した遠隔作業支援システムの提案}{-問題志向型システム(POS)をベースとしたノウハウのデータ化のためのシステムモデル-}
\etitle{aaaaaaaaa}{}

\nendo{2018}

\thesistype{博士論文}
\belongdept{工学研究科 工学専攻}
\belongcourse{千葉工業大学 大学院}

\ebelongdept{Graduate School of Engineering, Engineering major}
\ebelongcourse{Chiba Institute of Technology}

\authora{菅野~翔平}{}{Shohei~Sugano}
%
\datestamp{2023}{3}
%
\headshorttitle{時間的非同期性を考慮した遠隔作業支援システムの提案}
%
%%%%%%%%%%%%%%%%%%%%%%%%%%%%%%%%%%%%%%%%%%%%%%%%%%%
% 絶対に読み込まなければならないスタイル
%  順番を他の位置にしないほうが良い.
%%%%%%%%%%%%%%%%%%%%%%%%%%%%%%%%%%%%%%%%%%%%%%%%%%%
% E-Sys用条件定義スタイルファイル 宮治作成
\usepackage{esyshdr}
% 
%%%%%%%%%%%%%%%%%%%%%%%%%%%%%%%%%%%%%%%%%%%%%%%%%%%
% プリアンブル終了
% ドキュメント開始(理解していない人は書き換えないように)
%%%%%%%%%%%%%%%%%%%%%%%%%%%%%%%%%%%%%%%%%%%%%%%%%%%
%
\makeglossary
\begin{document}
%
%%%%%%%%%%%%%%%%%%%%%%%%%%%%%%%%%%%%%%%%%%%%%%%%%%%
% タイトルページコマンド起動
%  受領印ページと表紙から構成
%%%%%%%%%%%%%%%%%%%%%%%%%%%%%%%%%%%%%%%%%%%%%%%%%%%
%
\esysmaketitle
%
%%%%%%%%%%%%%%%%%%%%%%%%%%%%%%%%%%%%%%%%%%%%%%%%%%%
% 本文前のページ設定
%  論文要旨は youshi.tex,ABSTRACTは abstract.tex,
%  謝辞は thanks.tex に記述すること
%  anstract.tex 内は所定のフォーマットにて記述すること
%  ゼミレポ作成は全てコメントアウト卒論作成時に戻す     %%%%modified by sugacchi 2012. June 21
%%%%%%%%%%%%%%%%%%%%%%%%%%%%%%%%%%%%%%%%%%%%%%%%%%%
%
\pagenumbering{roman}%ココはコメントアウトしない
%
%%%%%%%%modified by sugacchi 2012. June 21
%
\chapter*{謝辞}
\addcontentsline{toc}{chapter}{謝辞}
%==============================================================
%========================= 謝辞 ===============================
%==============================================================

aaa


\clearpage
\chapter*{論文要旨}
\addcontentsline{toc}{chapter}{論文要旨}
%==============================================================  
%=================  論文要旨(日本語) ========================
%==============================================================

本研究では,.\\
結果として,....
\par
\par
本論文では,まず,....
\par
次に,....
\par
さらに,....
\clearpage
\chapter*{ABSTRACT}
\addcontentsline{toc}{chapter}{ABSTRACT}
\setlength{\parindent}{7ex}
%==============================================================
%=================  �_���v�|�@�i�@�p��@�j ========================
%==============================================================
\noindent In this research, I developed an excretion detection device using sensor technology.  

\setlength{\parindent}{1em}
\clearpage
%
%%%%%%%%%%%%%%%%%%%%%%%%%%%%%%%%%%%%%%%%%%%%%%%%%%%
% 目次ページ設定
% 目次,図目次,表目次,記号一覧
%%%%%%%%%%%%%%%%%%%%%%%%%%%%%%%%%%%%%%%%%%%%%%%%%%%
%
\tableofcontents
\addcontentsline{toc}{chapter}{目次}
\clearpage

\listoffigures
\addcontentsline{toc}{chapter}{図目次}
\clearpage

\listoftables
\addcontentsline{toc}{chapter}{表目次}
\clearpage

\printglossary
\addcontentsline{toc}{chapter}{記号一覧}
\clearpage

\pagebreak
\pagenumbering{arabic}
\pagestyle{fancy}

%
%%%%%%%%%%%%%%%%%%%%%%%%%%%%%%%%%%%%%%%%%%%%%%%%%%%
% 本文を開始
% ここから下を各自の設定にあわせて変更
%%%%%%%%%%%%%%%%%%%%%%%%%%%%%%%%%%%%%%%%%%%%%%%%%%%
%
\chapter{���_}\label{chap:1}
�{�͂ł́C�{�����̔w�i����іړI�C�܂��C�{�_���͍̏\���ɂ‚��ďq�ׂ�D



%1.1 �����w�i
%1.1
\section{�����w�i}\label{chap:1.1}
�����ł́C�{�����̔w�i�ɂ‚��ďq�ׂ�D


\subsection{�l�����ڂƍ����}\label{chap:1.1.1}
aa



%1.2 �����p��̗��j
%\input{main/chap_1/1.2_history}


%1.3 �J�����@�_�ɂ‚���
%\input{main/chap_1/1.3_devmethod}


%1.4 P-mSHELL���f���ɂ‚���
%\input{main/chap_1/1.4_PmSHELL}



% \chapter{���Ɩ��ɂ‚��Ă̒���}\label{chap:2}
�{�͂ł́C���{�b�g�Z�p�̓��������߂��Ă�����Ɩ��̌��؂̂��߁C��쌻��ŏ]��������E��ΏۂɋƖ��ʂ̎��ԕ��z�ɂ‚��ẴA���P�[�g���s�����D
�A���P�[�g���ʂ��C�����̎��Ə��Ŕr���Ɩ��Ɏ��Ԃ���₳��Ă��邱�Ƃ����炩�ɂȂ����D
�܂��C����ɔ����r���Ɩ��Ɋւ��鐸�_�I���S��P�A�̎�������s�������Ƃ��Ɩ��ߑ��̂��߁C�\���ɍs���Ȃ����Ƃ����炩�ɂȂ����D

%2.1 �A���P�[�g����
%2.1
\section{�A���P�[�g����}\label{chap:2.1}
�{�߂ł̓A���P�[�g�Ƃ��̌��ʂɂ‚��ďq�ׂ�D

\subsection{�A���P�[�g�����̎��{���@}\label{chap:2.1.1}
���{�b�g�Z�p�̓��������߂��Ă�����Ɩ��̌��؂̂��߁C��쌻��ŏ]��������E��ΏۂɁC
���{�ݎ�ʂ��ƂɋƖ��ʂ̎��ԕ��z�ɂ‚��ăA���P�[�g���s���C�ǂ̋Ɩ��Ɉ�ԁC�J�͂Ǝ��Ԃ���₳��Ă��邩�𒲍������D
�i�{�����́C����24�N�x�̌o�ώY�Əȁu�V���Ƒn�o�̂��߂̖ڗ����E�x���l�ވ琬�����Ɓv�ɂĎ��{�������̂ł���.�j
�A���P�[�g�͊y�V���T�[�`��p��Web�A���P�[�g�Ƃ���2013�N6���Ɏ��{�����D
�A���P�[�g���ڂ́C�I����232���ځC���R�L�q����10���ڂł���D
�A���P�[�g�ɂ͑S����692�l���񓚂����D
�i�A���P�[�g���ڂƃA���P�[�g�������ʂ�\ref{questionnaire}���Q�ƁD�j


\subsection{���Ɩ��̒�������}\label{chap:2.1.2}
���s��S�̂ɂ�����C�r����삪�K�v�ȗv���҂͖�278���l�i�S�̂�47\%�j�ł���D
���̒��ł��C�u�Q������ł��ނŽg�p�v�̗v���҂͖�122���l�i�S�̂�20\%�j�ɂ̂ڂ�D
�r����삪�K�v�ȗv���Ґ���C�u�Q������ł��ނŽg�p�v�̗v���҂́C�Ƃ��ɍݑ�ʼn����󂯂Ă�����������ł������D
�r����삪�K�v�ȗv���҂̂����C�{�ݗ��p�҂͖�101���l�i36\%�j�ł���C�ݑ���͖�177���l�i64\%�j�ł������D
�u�Q������ł��ނŽg�p�v�̂����C�{�ݗ��p�Ґ��͖�49���l�i40\%�j�C�ݑ���͖�73���l�i60\%�j�ł������D
\par
����́C�{�݉����󂯂��Ă���v���҂����C�ݑ�ʼn����󂯂Ă���v���҂̕����l���Ƃ��đ傫�����Ƃ�\���Ă���D
�܂��C�{�݉��ł́C���ی��{�݋y�э��z�ȘV�l�z�[���̕����u�Q������ł��ނŽg�p�v�̊������������Ƃ��킩�����D
\par
���ی��{�݁i�O��j�y�э��z�ȘV�l�z�[���ł́C�{�ݗ��p�҂̖�4���i��39���l�j���u�Q������ł��ނŽg�p�v�ɏ�����D
����ŁC��z�ȘV�l�z�[���C�O���[�v�z�[���y�я��K�͑��@�\�{�݂ł́C��2�����x�i��10���l�j�ł������D
�ݑ���i�u���T�[�r�X�𗘗p���Ă��Ȃ��v���܂ށj�̖�5�����ʏ����ł�����̂́C�K���엘�p�ƒ낪�u�Q������ł��ނŽg�p�v������25\%�ƍł��������Ƃ��킩�����D
�ʏ����̎{�ݗ��p�҂̖�1���i��29���l�j���u�Q������ł��ނŽg�p�v�ł������D
�K����ł́C�u�Q������ł��ނŽg�p�v�̊�����25\%�i��37���l�j�ɂ��̂ڂ�(\refFigJp{fig:2.1.2})�D
\par
���̌��ʂ���C�{���i�̒��߃^�[�Q�b�g���[�U�́C�u�Q������C���C���ނŽg�p�v�̍���҂ɍi�邱�ƂƂ����D
����ɁC�{���i���x�b�h��Ŏg�p���鐻�i�ł��邱�Ƃ���C�x�b�h��ʼn߂������Ԃ������u�Q������C���C���ނŽg�p�ҁv�𒼋߂̃^�[�Q�b�g�Ƃ��邱�ƂƂ����D
�v���҂̐l���Ƃ��Ă͍ݑ���̕����������C�ݑ���̌���́C�{�݉��̌���ɔ�ׂāC���|������I�ɍs���Ă��Ȃ��D
����䂦�C�������̂̐����L�������Ȃ�C�{���i���댟�m����”\�����������D
\par
���ۂɁC�J�������ɊJ���֌W�҂̎���Ŗ{���i�̎���@���g�p�����Ƃ���C�g�p���Ԓ��ɂ����ăZ���T�l������ɒB���Ă������Ƃ��������D
����́C���̊J���҂̎���ŔL�������Ă������߁C�L�̂ɂ����ɉߏ�ɔ������Ă��܂������Ƃ��l������D
\par
�ȏ�̗��R���C�{���i�̒��߃^�[�Q�b�g���[�U�́u�Q������C���C���ނŽg�p�v�̍���҂ɍi�����D
�܂��C�g�p����͉��{�݂Ɍ��肷�邱�ƂƂ����D

\begin{figure}[H]
    \centering
    \includegraphics[width=12cm]{fig/for_appendix/0701_aba�s�ꕪ��/�X���C�h4.eps}
    \caption{�{���i�̃��[�U�ɂȂ肤��Ώێ҂̊����� (�t�^�Q��)) }
    \label{fig:2.1.2}
 \end{figure}





% \chapter{�J��}\label{chap:3}
�{�͂ł́C�z��������уZ���T�����ō\�������n�[�h�E�F�A�C�Z���T�����r���L�C�̉�͂��s���\�t�g�E�F�A�ɕ�����邽�߁C�ŏ��Ƀn�[�h�E�F�A�ɂ�����z�����̊J������уZ���T�����̊J�����������D���̌�C�Z���T������͂��s���\�t�g�E�F�A�̐���������D

% 3.1 �z�����̊J��
%3.1
\section{�z�����̊J��}\label{chap:3.1}
�܂��r�����m�V�X�e���̋z�����J���ɂ‚��ċL�q����D
\par
�{�����ň����r�����m�V�X�e���́C�x�b�h���̋�C���z������z������L����D
�z�����̓x�b�h�����̋�C�����ʋz���������Ă���C����ɂ���Ĕr�������ۂ̔r���L���z���C➑̓��ɂ���K�X�Z���T���Z���V���O����.
�܂��C�z�����͎d�l��C��ɗv���҂��\�����ɕ~���ꂽ��ԂɂȂ邽�߁C�����ԏA�Q���Ă��Ă��C�\���ɉߓx�̑̈���������Ȃ��݌v�����߂���D����͎g�p�҂̐Q�S�n�����łȂ��C�ߓx�̑̈��ɂ����ጁi���傭�����j�𖢑R�ɖh�����߂ł�����D
\par
����ɁC�z�����͓���I�ɔr�������ʼn��������”\�����\�����邽�߁C➑̕��Ƌz�����̗e�ՂȒE���C�z�����̃V�[�g��򂪉”\�ł���K�v������D
\par
�����̗v�_�𓥂܂��āC�z�����̊J�����s�����D�{�߂ł́C�ȏ�̗v�_�𓥂܂�����ŁC�ǂ̂悤�ɋz�����̉��ǂ��s���������q�ׂĂ����D

\subsection{�}�b�g���X���ߍ��݌^�z�����̊J��}\label{chap:3.1.1}
�����ł́C�}�b�g���X�֖��ߍ��񂾋z�����ɂ‚��Đ�������D
\par
�����C�z�����̍\�z�̓}�b�g���X���̂����蔲���C���蔲������Ԃ�➑̕����̂��̂𖄂ߍ��ލ\�z�������D����̓Z���T���Ȃǂ̋@�B���������[�U�̖ڂɐG��Ȃ��悤�ɂ��C���ƒ}�b�g���X��~���ΐݒu�����Ƃł��邱�Ƃ�ڎw�������߂ł���D
\par
�������C�}�b�g���X�̑I��͗v���҂̐g�̏󋵂ɍ��킹�ēK�X�•ʂɑI�肳��Ă���D�Ⴆ�Ύ��g�ŐQ�Ԃ肪�ł��Ȃ��v���҂ɂ́C�����ŐQ�Ԃ������G�A�}�b�g���X��p����ȂǁC�v���҂̏�Ԃɍ��킹�Ă���D
����Ƀ}�b�g���X�̓x�b�h�w�����C�������̓x�b�h�̃����^�����ɍ��킹�đI�肳���D
\par
���̂��߁C�r�����m�V�X�e���𓱓��������ɂ��ւ�炸�C�}�b�g���X�ɔr�����m�V�X�e�������ߍ��܂�Ă���ƁC�{�ݑ��͊����̃}�b�g���X�̏��������Ȃ��Ă͂Ȃ�Ȃ��Ȃ��Ă��܂��D���ʓI�ɁC�R�X�g�ʂ̑����ɂ��{�݂ւ̕��S�ɂȂ��Ă��܂����Ƃ��킩�����D
\par
�ȏ�̗��R����C�r�����m�V�X�e���̓}�b�g���X�ւ̖��ߍ��݂ł͂Ȃ��C�}�b�g���X��ɕ~���V�[�g�^�̍\�z�ɕύX�����D


\subsection{�V���R���ɂ��V�[�g�J��}\label{chap:3.1.2}
�����ł́C�V���R���f�ނ��g�p�����z���V�[�g�̊J���ɂ‚��ďq�ׂ�D
�O�q�̒ʂ�C�}�b�g���X����➑̑S�Ă������邱�Ƃ͒f�O���C�V�[�g�^�ł̃v���g�^�C�v�J����V���ɍs�����ƂƂ����D���̃^�C�v�ł���΁C�ǂ̂悤�Ȏ�ނ̃}�b�g���X�ɂ��~�����Ƃ��”\�ł���C��̌��O�������ł���Ɣ��f�����D
�V�[�g�̑f�ނ�I�肷��ۂɂ́C�ȉ���2�_�ɗ��ӂ��đf�ޑI����s�����D

\begin{itemize}
\item �̈����U��
\item �q����
\end{itemize}

�܂��̈����U���ɂ‚��āC3.1�ł��O�q�����悤�ɁC�A�Q���ɘA�����Ďg�p���邽�߁C�̂ւ̈����ɒ[�ɂ�����Ȃ��K�v���������D
���ɁC�q���ʂɂ‚��Ĕr�����Ȃǂɏ펞���������”\�������邽�߁C�����ւ̔z�����K�v�ł������D���̂��߁C��򂪔�r�I�e�Ղł���C�”\�ł���΍R�ۍ�p������f�ނ�I�肵���D
\par
���ʂƂ��āC��L�̏����𖞂������̂Ƃ��ăV�[�g�f�ނɃV���R����I�肵��(\refFigJp{fig:3.1.2})�D
�V���R���͍d�x�����R�ɕς��邱�Ƃ��ł��C�z�����ɓK�����_���Nj����邱�Ƃ��”\�ł͂Ȃ����ƍl�����D�܂��C�V���R�����̂��̂ɍR�ۍ�p������C���C�V���R���{�̂����e�Ղł��������ߍ̗p���邱�ƂƂ���\refFigJp{fig:3.1.2}�D
�������C�V���R���f�ނł͈ȉ��̓_�ɖ�肪�������D

\begin{itemize}
\item �����Ԃ̎g�p�ɕs����
\item �V�[�g�̐��p�x���������߁C�������ꂽ�ߗނȂǂƈꏏ�ɐ􂦂Ȃ��D
\end{itemize}

��–ڂ́C�̂ւ̋ɒ[�Ȉ��͔�����ꂽ���C�����ԃV���R�����̋z�����ɉ�����邱�Ƃ��\���̒ɂ݂�Q�S�n�������A�Q�ɉe�����o��Ȃǂ��N�����D
�܂��C�����Ԏg�p���邱�ƂŔ��̎ア����҂ł������ጌ����ɂȂ�”\���������ƍl����ꂽ�D
��–ڂ́C�\�z�ȏ�ɐ��p�x�������C�������ꂽ���̈ߗނȂǂƈꏏ�ɐ���ł���Ȃ�, �������̏���������������K�v���������D
\par
�ȏ���C�V���R���f�ނł̋z�����J����f�O���C���_�������C���C����”\�ȕz�f�ނ�p�����V�[�g���J�����邱�ƂƂ����D
�z�����̍ŏI���i�ɂ‚��ẮC�{���i�̋����J���Ђł���p���}�E���g�x�b�h������Ђ̓����i����2018-066885�j�ƂȂ��Ă��邽�߁C��������Q�Ƃ��ꂽ���D

\begin{figure}[H]
   \centering
   \includegraphics[width=8cm]{fig/�V���R���f�ނō쐬�����V�[�g����.eps}
   \caption{�V���R���f�ނō쐬�����V�[�g����}
   \label{fig:3.1.2}
\end{figure}


% 3.2 ➑̕��̊J��
\input{main/chap_3/3.2_sensorunit}

% 3.3 �A���S���Y���̍\�z
\input{main/chap_3/3.3_algorithm}

% 3.4 ���i�̌��؎���
\input{main/chap_3/3.4_experiment}

% 3.5 Web�A�v���̊J��
\input{main/chap_3/3.5_devwebapp}

% 3.6 Web�A�v���̌��؎���
\input{main/chap_3/3.6_webappexpt}

% \chapter{���i�̉��l���L�ɂ‚���}\label{chap:4}
�{�͂ł́C�J���ґ����z�肵�����i�̉��l���C���[�U�[�ɑ΂��C���₩�ɐ��������l���L���s�����߂ɍs�������l���L��@�ɂ‚��ďq�ׂ�D
�܂��C���l���L�̈Ӌ`�ɂ‚��ďq�ׂ�D
���ɁC���i���l�̐����Ƌ��L���@�̖͍��Ƃ��āC���i���̕ύX��^�O���C���̍쐬�C���i�Љ��̐���ɂ‚��ďq�ׂ�D

% 4.1 ���l���L�̈Ӌ`
%4.1
\section{���l���L�̈Ӌ`}\label{chap:4.1}
�����p��̐��i�J���ɂ����āC���l���L���ӎ��I�ɍs���K�v�����闝�R�͈ȉ��ł���D

\begin{enumerate}
\item ��쌻��ɂ͕����̃X�e�[�N�z���_�����݂���D
\item �X�e�[�N�z���_�ɂ���Đ��i�ւ̉��l�����قȂ�D
\end{enumerate}

��–ڂ́C��쌻��ɂ͕����̃X�e�[�N�z���_�����݂��邽�߂ł���D
���T�[�r�X���󂯂�v���҂́C�{���i���ӎ��I�Ɏg�p���邱�Ƃ͂Ȃ����C�{���i�̌��ʂ����󂷂��v�҂ł���D
���T�[�r�X��񋟂�����҂́C�{���i���ӎ��I�Ɏg�p����g�p�҂ƂȂ�D�{���i���r�������m�����ۂɂ͎��ۂɂ��ނŒ�����r���L�^���s���C�r���L�^����Z�o����r���p�^�[���\�𗘊��p����̂��C�g�p�҂ł���D
�܂��{���i�̍w���ӎv����́C��ɉ��{�ݒ��Ȃnjo�c�҂��s�����߁C�o�c�҂������I�ȍw���҂ł���D\par
��–ڂ́C�X�e�[�N�z���_�������l���݂��邽�߁C�{���i�ɑ΂��鉿�l�́C�X�e�[�N�z���_���ꂼ��̗���ɂ���ĈقȂ�D
��v�҂ł���v���҂́C�{���i�ɂ���Ď������g���󂯂�r���P�A�₨�ނŒ����^�C�~���O���K��������邩���d�v�ł���D
�{���i�̎g�p�҂ł�����҂́C�{���i���g�p����ۂ̓��̓C���^�[�t�F�[�X�̎g���₷����C�f�[�^�����p�̂��₷���C�܂��r�����ɂ�鉘�����̐��̂��₷���Ȃǂ��d�v�������D
���������҂ɂƂ��āC�{���i�ɂ���ėv���҂ɑ΂���P�A�̌��オ�s���邩�ۂ����d�v�����C�r���P�A���̂̌���́C�{���i�P�Ƃł͒񋟂ł��Ȃ����ƁC�܂��g�p�҂ł�����҂ɂƂ��ẮC���X���i�Ɛڂ��钆�ŁC�X�g���X�Ȃ��g�p�ł��邩�͏d�v�ȓ_�ƂȂ�D\par
�w���҂ł���o�c�҂́C�{���i�𓱓����邱�ƂŐl����₨�ނ—ނȂǂ̏��Օi���ǂꂾ���팸�ł���̂��C�������́C�{���i�𓱓����邱�ƂŐE���ɑ΂��鏈�����P��ڋq�����x���P���ǂꂾ���s���邩�C����ɂ́C��i�I�Ȏ��g�݂����Ă��鎖�Ə��Ƃ��đΊO�I�ɐ�`�”\�ƂȂ邱�ƂŁC�l�ނ̗̍p�����グ���邩�Ȃǂ��d�v�������D\par
�܂��C�{�݌o�c�҂́C�K��������쌻��ɂ����邨�ނŒ����o����r���P�A�o����L���Ȃ����߁C�{���i�����Ɩ��ɂǂ̂悤�Ȍ��ʂ������炷�̂��C��̓I�ɑz�����邱�Ƃ�����ꍇ������D
���̂��߁C�{�݌o�c�҂֖{���i�̉��l���������ۂɂ́C�o�c�ʂɂ����郁���b�g�����łȂ��C�{���i�����Ɩ��ɂ����Ăǂ̂悤�ɗ����p�����̂����C�”\�Ȍ����̓I�ɐ�������K�v������(\refFigJp{fig:4.1})�D
\par
�ȏ���C�{���i�̉��l��i������l�̗����C���Ɩ��ɑ΂��闝��x�ɂ���āC���i���l�̌��o�������قȂ�ƍl���Ă����D
����ɂ��C�{���i�ɂ����Ă͎�v�҂Ǝg�p�҂ƍw���҂̗��ꂪ�݂��ɈقȂ�C���C���Ɩ��ւ̗���x����l�ł͂Ȃ��Ƃ��C�{���i�ɑ΂��鉿�l���L���s����悤�z�����C���i���������Ȃǂ��쐬���邱�ƂƂ����D
���߂ł́C�{���i�̉��l���L���ǂ̂悤�ɍs���������q�ׂ�D

\begin{figure}[H]
   \centering
   \includegraphics[width=12cm]{fig/�e�X�e�[�N�z���_�̎v��.eps}
   \caption{�e�X�e�[�N�z���_�̎v��}
   \label{fig:4.1}
\end{figure}




% 4.2 ���i���l�̐����Ƌ��L���@�̖͍�
\input{main/chap_4/4.2_sharing}


% 4.3 �����p�������l���L�̗L�p��
\input{main/chap_4/4.3_movieshare}

% \chapter{���_}\label{sec:9}
�{�͂ł́C���_����і{���i�̍���̓W�]�ƌ��_�ɂ‚��ďq�ׂ�D

�{���i�́C��쌻��ɂ����邨�ނŒ����Ɩ��̓K������ړI�Ƃ��ĊJ�����ꂽ�D
���̂��߂ɁC�r�����̏L�C����r�����m���邽�߂̋�C�z������➑̂̊J���C�r�����m���s�����߂̔r�����m�A���S���Y���J���C
�r���f�[�^�����p�����r���p�^�[���\�𐶐�����Web�A�v���̊J���C�e�X�e�[�N�z���_�ւ̓K�؂ȉ��l���L��i�̍\�z���s�����D
�����̐��ʂɂ��C����܂ŗv���҂̔r���^�C�~���O�ɍ��킹�����ނŒ������s���Ă��Ȃ������{�݂ł��C�K�؂ȃ^�C�~���O�ł̂��ނŒ������”\�ƂȂ�D
����ɂ��CP-mSHELL���f���Œ�`�����C���V�f���g�͉��P�����”\�������܂�C���ނŒ����Ɩ��̓K�����C�񋟂ł���r���P�A�̌��オ�”\�ƂȂ�C
�ԐړI�ɂ͉��E�̏������P�C�v���҂�QOL�S�̂̌���C�{�݌o�c�ɂ�����ڋq�����x�̌���Ȃǂ��l������D


% 5.1 �܂Ƃ�
\input{main/chap_5/5.1_summary}


% 5.2 ����̓W�]
\input{main/chap_5/5.2_perspect}

%
%%%%%%%%%%%%%%%%%%%%%%%%%%%%%%%%%%%%%%%%%%%%%%%%%%
% 本文はここまで.カウンタをいったん吐き出す.
%%%%%%%%%%%%%%%%%%%%%%%%%%%%%%%%%%%%%%%%%%%%%%%%%%
%
\clearpage
%
%%%%%%%%%%%%%%%%%%%%%%%%%%%%%%%%%%%%%%%%%%%%%%%%%%%
% 脚注(後注)をつける
%  本文中に脚注(\footnote)をつけなかった場合は,
%  コメントアウトしてください.
%%%%%%%%%%%%%%%%%%%%%%%%%%%%%%%%%%%%%%%%%%%%%%%%%%%
%
%\listnotesname
%\addcontentsline{toc}{chapter}{後注}
%
%%%%%%%%%%%%%%%%%%%%%%%%%%%%%%%%%%%%%%%%%%%%%%%%%%%
% 参考文献をつける
%  bibtex のスタイルファイル(bstファイル)には
%   esysthesis.bst を指定
%  \bibliographyの引数には参考文献ファイルを指定 
%  参考文献ファイル(bibファイル)が複数存在する場合には,
%   WelfareJ,WelfareTのように半角カンマで区切って指定する.
%%%%%%%%%%%%%%%%%%%%%%%%%%%%%%%%%%%%%%%%%%%%%%%%%%%
%
\bibliographystyle{esysthesis}
%\begin{flushright}
\begin{flushleft}
    \bibliography{bib/reference}
\end{flushleft}
%\end{flushright}
\addcontentsline{toc}{chapter}{参考文献}
%
%%%%%%%%%%%%%%%%%%%%%%%%%%%%%%%%%%%%%%%%%%%%%%%%%%%
% 付録環境の設定
%  付録には,最低でも各自が作成したプログラムやデータの
%  利用方法に関する解説および研究発表時に指摘された
%  質疑応答を載せること.
%%%%%%%%%%%%%%%%%%%%%%%%%%%%%%%%%%%%%%%%%%%%%%%%%%%
%
%%%%%%%%%%%%%%%%%%%%%%%%%%%%%%%%%%%%%%%%%%%%%%%%%%%
% この部分は修正してはいけない
%%%%%%%%%%%%%%%%%%%%%%%%%%%%%%%%%%%%%%%%%%%%%%%%%%%
%
\appendix
%
%%%%%%%%%%%%%%%%%%%%%%%%%%%%%%%%%%%%%%%%%%%%%%%%%%%
% ここから下に付録の内容を書いていくこと
%%%%%%%%%%%%%%%%%%%%%%%%%%%%%%%%%%%%%%%%%%%%%%%%%%%
%
%%%%%%%%%%%%%%%%%%%%%%%%%%%%%%%%%%%%%%%%%%%%%%%%%%%
% CD-ROMとして添付するプログラムおよびデータの利用方法
%  appendix.tex に記述(変更可)
%%%%%%%%%%%%%%%%%%%%%%%%%%%%%%%%%%%%%%%%%%%%%%%%%%%
%
% \chapter{�{�����̃f�[�^}\label{chap:A}
�����ł́C�{�����ł̎������L�ڂ���B


\section{�������̏ڍׂȎ菇}\label{experiment_process}
\begin{verbatim}
(1) 18�F30 �v���O�����̓���m�F(�����[�g)���s���D
(2) 19�F00 ����₩���ɓ�����C�����Ɏ��̎��������{����D
  [1] �����@��̈ȉ��̓_�����s���D
    1.PC��@�킪���삵�Ă��邩�m�F(��肪����΁Cslack�ŘA�����邱��)
    2.�S���̃`���[�u�̐�����
  [2] 19�F00�̒莞���ނŒ����^�팱�ґS���̂��ނ“����m�F���C�ȉ������{����D
    1.���ނ‚�����Ă���ꍇ�^���ނŒ��������{����D
    2.�e�팱�҂̂��ނ‚̓��ĕ����C�w���ʂ�̓��ĕ��ƈقȂ��Ă���ꍇ�^�w���ʂ�̓��ĕ��ɒ����D
    3.���ނ‚�����Ă��Ȃ��ꍇ�^���ނ“��ɔr�������������Ƃ��m�F���āC�Ăт��ނ‚�‚���D
  [3] ��ƏI����C�u�X�^�b�t�X�e�[�V������PC�Ɍ��ʂ��L�^�v����D
����) [2]�ōs����19�F00�̒莞���ނŒ����́C(3)�̐������ނŒ����Ɋ܂܂�Ȃ��D
(3) �������ނŒ����^�r�����m�Z���T�̒ʒm����������C�ȉ������{����D
  [1] �r���̒ʒm����������C�����ɔ팱�҂̂��ނ“����m�F����D
  [2] �r����������΁C���ނŒ��������{����D�r�������Ȃ���΁C�Ăт��ނ‚�‚���D
  [3] �r����������΁C���̗�(�d��)�𑪒肷��(���ꂽ�I���c�d�ʁ|���ꂢ�ȃI���c�d��)
  [4] ��ƏI����C�u�X�^�b�t�X�e�[�V������PC�Ɍ��ʂ��L�^�v����D
����) �ʒm���������Ƃ��Ă��C�r�����������������́C�������ނŒ������������ƂɊ܂܂�Ȃ��D
(4) �莞�̑̈ʌ����^����₩���Ō��߂�ꂽ�^�C���X�P�W���[���ɏ]���āC�e�팱�҂̑̈ʌ��������{����
(5) 01�F00�̒莞���ނŒ����^(3)�̐������ނŒ�������x���s���Ă��Ȃ��팱�҂ɑ΂��ẮC���L�̗v�̂Œ莞���ނŒ��������{����D
  [1] ���ނ‚�����Ă���ꍇ�^���ނŒ��������{���C�r�����̗�(�d��)�𑪒肷��D
  [2] ���ނ‚�����Ă��Ȃ��ꍇ�^���ނ“��ɔr�������������Ƃ��m�F���āC�Ăт��ނ‚�‚���D
  [3] [1][2]�Ƃ��ɁC��ƏI����C�u�X�^�b�t�X�e�[�V������PC�Ɍ��ʂ��L�^�v����D
  �� �莞���ނŒ������{���ɁC�ʂ̔팱�҂Ŕr���̒ʒm���������ꍇ�C���ݍs���Ă��邨�ނŒ������I������C�r���̒ʒm���������팱�҂̂��ނ“��̊m�F((3)�̍��)��D�悵�čs������
(6) (5)���I����C(3)(4)���p������D
(7) 05�F00���̒莞���ނŒ����^�팱�ґS���ɑ΂��āC�莞�̂��ނŒ��������{����D��Ƃ̗v�̂�(5)��[1]�`[3]�Ɠ����ł���D
(8) 05�F30 �����̏I����Ƃ����{����D
  [1] �{���̋L�^�������S�āu�X�^�b�t�X�e�[�V������PC�Ɍ��ʂ��L�^�v����Ă��邱�Ƃ��m�F���C�t�@�C�����㏑���ۑ�������ŃV���b�g�_�E������D
  [2] ����3F�̃X�^�b�t�X�e�[�V�����ɒu���Ă���^�C���J�[�h��؂�C���̎����̉��Ɏ����̖��O���L������D
  [3] ����₩���̐E���̕��ɁC�A���|��`���Ă���C�ފق���D
  �� �팱�҂̎���ɐݒu����Ă���r�����m�Z���T�₻��ɂ‚Ȃ����Ă���PC�̓d���́C�‚����܂܂�OK

�X�^�b�t�X�e�[�V������PC�Ɍ��ʂ��L�^������e(�L�^�\.xlsx)
���ꂼ��̔팱�҂ɑ΂��āC�ȉ��̎�����PC�̎w��̃t�@�C���ɋL�^����
(a) �r���̒ʒm�̂��������ԁE�E�E(3)(6)�̍��
(b) ���ނŒ�����Ƃ��J�n�������ԁE�E�E(2)(3)(5)(6)(7)�̍��
(c) ���ނŒ�����Ƃ��I���������ԁE�E�E(2)(3)(5)(6)(7)�̍��
(d) ���ނ“��̔r�����̏󋵁E�E�E(2)(3)(5)(6)(7)�̍��
  [1] �r�����̎�ށ^�A�C�ցC�A���ցC�Ȃ�
  [2] �r�����̗� �^���肵���r�����̏d��
  [3] �ւ̏ꍇ�C���̏�ԁ^�u���X�g���X�P�[���ɂđI��
(e) �̈ʌ������s�������ԁE�E�E(4)
(f) �̈ʌ����̌����^�̈ʌ����̑O�̌�������̌����E�E�E(4)
(g) ���̑��C���M����(�C��)�Ƃ���ɋC�Â�������
  ��F�������ɁC�u�֘R�ꂪ�������C�V�[�g�����ꂽ�v�C�u�@�B���~�܂��Ă����̂ɋC�Â����v�C�u�팱�҂��������Ă����v�Ȃ�
  �� ���Ԃ́C���������ƒX�}�z�ȂǁC�d�g���v�̎����Ŋm�F���邱�ƁD�{�݂̎��v�́C����Ă���”\������
  �� �x�b�h�T�C�h�֋L�^�p���������čs���C�܂��͂��ނŒ����̍�Ƃ̏�Ŏ菑���ŋL�^���Ƃ�C�e��Ƃ̏I����C�󂢂����ԂŃX�^�b�t�X�e�[�V������PC�Ɍ��ʂ���͂��邱�Ƃ𐄏�����D
\end{verbatim}


\clearpage
\section{�s�ꕪ�͎���}\label{market_analysis}

\begin{figure}[p]
  \centering
  \includegraphics[angle=90, width=15cm]{fig/for_appendix/0701_aba�s�ꕪ��/�X���C�h1.eps}
  \label{market_analysis_zu1}
\end{figure}
\begin{figure}[p]
  \centering
  \includegraphics[angle=90, width=15cm]{fig/for_appendix/0701_aba�s�ꕪ��/�X���C�h2.eps}
  \label{market_analysis_zu2}
\end{figure}
\begin{figure}[p]
  \centering
  \includegraphics[angle=90, width=15cm]{fig/for_appendix/0701_aba�s�ꕪ��/�X���C�h3.eps}
  \label{market_analysis_zu3}
\end{figure}
\begin{figure}[p]
  \centering
  \includegraphics[angle=90, width=15cm]{fig/for_appendix/0701_aba�s�ꕪ��/�X���C�h4.eps}
  \label{market_analysis_zu4}
\end{figure}
\begin{figure}[p]
  \centering
  \includegraphics[angle=90, width=15cm]{fig/for_appendix/0701_aba�s�ꕪ��/�X���C�h5.eps}
  \label{market_analysis_zu5}
\end{figure}
\begin{figure}[p]
  \centering
  \includegraphics[angle=90, width=15cm]{fig/for_appendix/0701_aba�s�ꕪ��/�X���C�h6.eps}
  \label{market_analysis_zu6}
\end{figure}
\begin{figure}[p]
  \centering
  \includegraphics[angle=90, width=15cm]{fig/for_appendix/0701_aba�s�ꕪ��/�X���C�h7.eps}
  \label{market_analysis_zu7}
\end{figure}
\begin{figure}[p]
  \centering
  \includegraphics[angle=90, width=15cm]{fig/for_appendix/0701_aba�s�ꕪ��/�X���C�h8.eps}
  \label{market_analysis_zu8}
\end{figure}
\begin{figure}[p]
  \centering
  \includegraphics[angle=90, width=15cm]{fig/for_appendix/0701_aba�s�ꕪ��/�X���C�h9.eps}
  \label{market_analysis_zu9}
\end{figure}
\begin{figure}[p]
  \centering
  \includegraphics[angle=90, width=15cm]{fig/for_appendix/0701_aba�s�ꕪ��/�X���C�h10.eps}
  \label{market_analysis_zu10}
\end{figure}
\begin{figure}[p]
  \centering
  \includegraphics[angle=90, width=15cm]{fig/for_appendix/0701_aba�s�ꕪ��/�X���C�h11.eps}
  \label{market_analysis_zu11}
\end{figure}
\begin{figure}[p]
  \centering
  \includegraphics[angle=90, width=15cm]{fig/for_appendix/0701_aba�s�ꕪ��/�X���C�h12.eps}
  \label{market_analysis_zu12}
\end{figure}
\begin{figure}[p]
  \centering
  \includegraphics[angle=90, width=15cm]{fig/for_appendix/0701_aba�s�ꕪ��/�X���C�h13.eps}
  \label{market_analysis_zu13}
\end{figure}
\begin{figure}[p]
  \centering
  \includegraphics[angle=90, width=15cm]{fig/for_appendix/0701_aba�s�ꕪ��/�X���C�h14.eps}
  \label{market_analysis_zu14}
\end{figure}
\begin{figure}[p]
  \centering
  \includegraphics[angle=90, width=15cm]{fig/for_appendix/0701_aba�s�ꕪ��/�X���C�h15.eps}
  \label{market_analysis_zu15}
\end{figure}
\begin{figure}[p]
  \centering
  \includegraphics[angle=90, width=15cm]{fig/for_appendix/0701_aba�s�ꕪ��/�X���C�h16.eps}
  \label{market_analysis_zu16}
\end{figure}
\begin{figure}[p]
  \centering
  \includegraphics[angle=90, width=15cm]{fig/for_appendix/0701_aba�s�ꕪ��/�X���C�h17.eps}
  \label{market_analysis_zu17}
\end{figure}
\begin{figure}[p]
  \centering
  \includegraphics[angle=90, width=15cm]{fig/for_appendix/0701_aba�s�ꕪ��/�X���C�h18.eps}
  \label{market_analysis_zu18}
\end{figure}
\begin{figure}[p]
  \centering
  \includegraphics[angle=90, width=15cm]{fig/for_appendix/0701_aba�s�ꕪ��/�X���C�h19.eps}
  \label{market_analysis_zu19}
\end{figure}
\begin{figure}[p]
  \centering
  \includegraphics[angle=90, width=15cm]{fig/for_appendix/0701_aba�s�ꕪ��/�X���C�h20.eps}
  \label{market_analysis_zu20}
\end{figure}
\begin{figure}[p]
  \centering
  \includegraphics[angle=90, width=15cm]{fig/for_appendix/0701_aba�s�ꕪ��/�X���C�h21.eps}
  \label{market_analysis_zu21}
\end{figure}
\begin{figure}[p]
  \centering
  \includegraphics[angle=90, width=15cm]{fig/for_appendix/0701_aba�s�ꕪ��/�X���C�h22.eps}
  \label{market_analysis_zu22}
\end{figure}
\begin{figure}[p]
  \centering
  \includegraphics[angle=90, width=15cm]{fig/for_appendix/0701_aba�s�ꕪ��/�X���C�h23.eps}
  \label{market_analysis_zu23}
\end{figure}
\begin{figure}[p]
  \centering
  \includegraphics[angle=90, width=15cm]{fig/for_appendix/0701_aba�s�ꕪ��/�X���C�h24.eps}
  \label{market_analysis_zu24}
\end{figure}
\begin{figure}[p]
  \centering
  \includegraphics[angle=90, width=15cm]{fig/for_appendix/0701_aba�s�ꕪ��/�X���C�h25.eps}
  \label{market_analysis_zu25}
\end{figure}
\begin{figure}[p]
  \centering
  \includegraphics[angle=90, width=15cm]{fig/for_appendix/0701_aba�s�ꕪ��/�X���C�h26.eps}
  \label{market_analysis_zu26}
\end{figure}
\begin{figure}[p]
  \centering
  \includegraphics[angle=90, width=15cm]{fig/for_appendix/0701_aba�s�ꕪ��/�X���C�h27.eps}
  \label{market_analysis_zu27}
\end{figure}
\begin{figure}[p]
  \centering
  \includegraphics[angle=90, width=15cm]{fig/for_appendix/0701_aba�s�ꕪ��/�X���C�h28.eps}
  \label{market_analysis_zu28}
\end{figure}
\begin{figure}[p]
  \centering
  \includegraphics[angle=90, width=15cm]{fig/for_appendix/0701_aba�s�ꕪ��/�X���C�h29.eps}
  \label{market_analysis_zu29}
\end{figure}
\begin{figure}[p]
  \centering
  \includegraphics[angle=90, width=15cm]{fig/for_appendix/0701_aba�s�ꕪ��/�X���C�h30.eps}
  \label{market_analysis_zu30}
\end{figure}
\begin{figure}[p]
  \centering
  \includegraphics[angle=90, width=15cm]{fig/for_appendix/0701_aba�s�ꕪ��/�X���C�h31.eps}
  \label{market_analysis_zu31}
\end{figure}
\begin{figure}[p]
  \centering
  \includegraphics[angle=90, width=15cm]{fig/for_appendix/0701_aba�s�ꕪ��/�X���C�h32.eps}
  \label{market_analysis_zu32}
\end{figure}
\begin{figure}[p]
  \centering
  \includegraphics[angle=90, width=15cm]{fig/for_appendix/0701_aba�s�ꕪ��/�X���C�h33.eps}
  \label{market_analysis_zu33}
\end{figure}
\begin{figure}[p]
  \centering
  \includegraphics[angle=90, width=15cm]{fig/for_appendix/0701_aba�s�ꕪ��/�X���C�h34.eps}
  \label{market_analysis_zu34}
\end{figure}
\begin{figure}[p]
  \centering
  \includegraphics[angle=90, width=15cm]{fig/for_appendix/0701_aba�s�ꕪ��/�X���C�h35.eps}
  \label{market_analysis_zu35}
\end{figure}
\begin{figure}[p]
  \centering
  \includegraphics[angle=90, width=15cm]{fig/for_appendix/0701_aba�s�ꕪ��/�X���C�h36.eps}
  \label{market_analysis_zu36}
\end{figure}
\begin{figure}[p]
  \centering
  \includegraphics[angle=90, width=15cm]{fig/for_appendix/0701_aba�s�ꕪ��/�X���C�h37.eps}
  \label{market_analysis_zu37}
\end{figure}

\clearpage
\section{�A���P�[�g����}\label{questionnaire}
\ref{chap:4.3}�ōs�����A���P�[�g�̌��ʂ��ȉ��Ɍf�ڂ���D

\begin{figure}[p]
  \centering
  \includegraphics[width=15cm]{fig/for_appendix/�A���P�[�g/1.eps}
  \label{questionnaire_zu1}
\end{figure}
\begin{figure}[p]
  \centering
  \includegraphics[width=15cm]{fig/for_appendix/�A���P�[�g/2.eps}
  \label{questionnaire_zu2}
\end{figure}
\begin{figure}[p]
  \centering
  \includegraphics[width=15cm]{fig/for_appendix/�A���P�[�g/3.eps}
  \label{questionnaire_zu3}
\end{figure}
\begin{figure}[p]
  \centering
  \includegraphics[width=15cm]{fig/for_appendix/�A���P�[�g/4.eps}
  \label{questionnaire_zu4}
\end{figure}
\begin{figure}[p]
  \centering
  \includegraphics[width=15cm]{fig/for_appendix/�A���P�[�g/5.eps}
  \label{questionnaire_zu5}
\end{figure}
\begin{figure}[p]
  \centering
  \includegraphics[width=15cm]{fig/for_appendix/�A���P�[�g/6.eps}
  \label{questionnaire_zu6}
\end{figure}
\begin{figure}[p]
  \centering
  \includegraphics[width=15cm]{fig/for_appendix/�A���P�[�g/7.eps}
  \label{questionnaire_zu7}
\end{figure}
\begin{figure}[p]
  \centering
  \includegraphics[width=15cm]{fig/for_appendix/�A���P�[�g/8.eps}
  \label{questionnaire_zu8}
\end{figure}
\begin{figure}[p]
  \centering
  \includegraphics[width=15cm]{fig/for_appendix/�A���P�[�g/9.eps}
  \label{questionnaire_zu9}
\end{figure}
\begin{figure}[p]
  \centering
  \includegraphics[width=15cm]{fig/for_appendix/�A���P�[�g/10.eps}
  \label{questionnaire_zu10}
\end{figure}

\clearpage
%
%%%%%%%%%%%%%%%%%%%%%%%%%%%%%%%%%%%%%%%%%%%%%%%%%%%
% その他の付録
%  その他の付録が有る場合には,以下の部分で
%  付録ファイル名を指定する
%%%%%%%%%%%%%%%%%%%%%%%%%%%%%%%%%%%%%%%%%%%%%%%%%%%
%
%\input{main/oboegaki}
%\clearpage
%\input{main/kinshi}
%\clearpage
%
%%%%%%%%%%%%%%%%%%%%%%%%%%%%%%%%%%%%%%%%%%%%%%%%%%%
% 質疑応答集
%  qa.tex 内に所定のフォーマットにて記述すること.
%  この部分自体は修正しなくて良い
%%%%%%%%%%%%%%%%%%%%%%%%%%%%%%%%%%%%%%%%%%%%%%%%%%%
%
%%==================================================================
%======================== 質疑応答集 ==============================
%==================================================================
\chapter{質疑応答集}
本付録では,実際に論文発表会にて出された質疑応答を載せる.
\vspace{1cm}

\begin{questions}
\question{王教授}
{人工無能と人工知能の違いは?}
\answer{本論文21ページ,「4.3.2 人工知能と人工無能」を参照ください.}

%\question{Elise教授(富山研)}
%{何故 コノ村ニハ 今 誰モイナイノ?} 
%\answer
%{其れは 昔 皆 死んじゃったからさ}
%
%\question{Elise教授(富山研)}
%{ジャ... 何故 昔 村人 皆 死ンジャッタノ?} 
%\answer
%{其れは 黒き 死の 病 のせいさ}
%
%\question{Elise教授(富山研)}
%{ジャ... 何故 ソノ森ノ 村ニハ 母子ハイタノ?} 
%\answer
%{其れは 或の【イド】が 呼んだからさ}
%
%\question{Elise教授(富山研)}
%{ジャ... 何故 【イド】ハ 何ノ為ニ 人ヲ呼ブノ?} 
%\answer
%{其れこそが 奴の本能だからさ}
\end{questions}


%\clearpage
%
%%%%%%%%%%%%%%%%%%%%%%%%%%%%%%%%%%%%%%%%%%%%%%%%%%%
% 研究・発表業績
%  学外で発表,展示,論文公開などを行った際には,
%   achievement.tex 内に所定のフォーマットにて記述すること
%  発表などを行わなかったときには,下記の部分をコメントアウト
%   すること
%%%%%%%%%%%%%%%%%%%%%%%%%%%%%%%%%%%%%%%%%%%%%%%%%%%
%
%\newpage
%\addcontentsline{toc}{chapter}{研究業績2}
%\chapter*{研究業績}
%
%\chapter{研究業績}
%%%%%%%%%%%%%%%%%%%%%%%%%%%%%%%%%%%%%%%%%%%%%%%
%%		achievement.tex			�����Ɛ�
%%		�w��\�⍑�ۉ�c���̏�(�w�O)�Ŕ��\�����ꍇ
%%		�R�R�ɋL�ڂ��邱��
%%		���\�������ɍ����ƍ��O�ŕ����邱��
%%%%%%%%%%%%%%%%%%%%%%%%%%%%%%%%%%%%%%%%%%%%%%
%%ex) 
%%\section{�恇��E-sys�w����}
%%
%%
�����ł́C�Q�l�܂łɂ���܂Ŕ��\���������Ɛт��ڂ���D

\section{�������\}
\subsection{���{�����H�w��킢���l�H��������킢���l�H����������2���N�L�O�V���|�W�E��}
���� �ĕ�, �x�R ��. �u �����ɂ�����u���킢���v�̌��� -Roomba�̓�����p�������킢���̕���- �v, pp. 15-18, 2012.

\subsection{��14�񊴐��H�w����}
���� �ĕ�, �x�R ��. �u �����ɂ����邩�킢���̌��� -Roomba�̓�����p�������킢���̕���-�v, B2-02, 2012.

\section{���ۉ�c}
\subsection{4th  International Conference on Applied Human Factors and Ergonomics (AHFE) 2012}
Shohei SUGANO, Haruna MORITA, and Ken TOMIYAMA. ``Study on Kawaii in Motion -Classifying Kawaii Motion using Roomba-," pp. 4498-4507, 2012.
%
%\newpage
%
%
%%%%%%%%%%%%%%%%%%%%%%%%%%%%%%%%%%%
% 自己紹介ページ
%%%%%%%%%%%%%%%%%%%%%%%%%%%%%%%%%%%
%\birthday{1988}{9}{8}
%\hometown{●●●●●}
%\high_school{高等学校}
%\nickname{}{}{}{}{}{}
%\comment{■■■}
%\esysmakepostscript

%
%%%%%%%%%%%%%%%%%%%%%%%%%%%%%%%%%%%%%%%%%%%%%%%%%%%
% 本文を終了
%%%%%%%%%%%%%%%%%%%%%%%%%%%%%%%%%%%%%%%%%%%%%%%%%%%
%
\end{document}

