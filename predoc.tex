%%%%%%%%%%%%%%%%%%%%%%%%%%%%%%%%%%%%%%%%%%%%%%%%%%%
% タイトルページコマンド起動
%  受領印ページと表紙から構成
%%%%%%%%%%%%%%%%%%%%%%%%%%%%%%%%%%%%%%%%%%%%%%%%%%%
%
\esysmaketitle
%
%%%%%%%%%%%%%%%%%%%%%%%%%%%%%%%%%%%%%%%%%%%%%%%%%%%
% 本文前のページ設定
%  論文要旨は youshi.tex,ABSTRACTは abstract.tex,
%  謝辞は thanks.tex に記述すること
%  anstract.tex 内は所定のフォーマットにて記述すること
%  ゼミレポ作成は全てコメントアウト卒論作成時に戻す     %%%%modified by sugacchi 2012. June 21
%%%%%%%%%%%%%%%%%%%%%%%%%%%%%%%%%%%%%%%%%%%%%%%%%%%
%
\pagenumbering{roman}%ココはコメントアウトしない
%
%%%%%%%%modified by sugacchi 2012. June 21
%
\chapter*{謝辞}
\addcontentsline{toc}{chapter}{謝辞}
%==============================================================
%========================= 謝辞 ===============================
%==============================================================

aaa


\clearpage
\chapter*{論文要旨}
\addcontentsline{toc}{chapter}{論文要旨}
%==============================================================  
%=================  論文要旨(日本語) ========================
%==============================================================

本研究では,.\\
結果として,....
\par
\par
本論文では,まず,....
\par
次に,....
\par
さらに,....
\clearpage
\chapter*{ABSTRACT}
\addcontentsline{toc}{chapter}{ABSTRACT}
\setlength{\parindent}{7ex}
%==============================================================
%=================  �_���v�|�@�i�@�p��@�j ========================
%==============================================================
\noindent In this research, I developed an excretion detection device using sensor technology.  

\setlength{\parindent}{1em}
\clearpage
%
%%%%%%%%%%%%%%%%%%%%%%%%%%%%%%%%%%%%%%%%%%%%%%%%%%%
% 目次ページ設定
% 目次,図目次,表目次,記号一覧
%%%%%%%%%%%%%%%%%%%%%%%%%%%%%%%%%%%%%%%%%%%%%%%%%%%
%
\tableofcontents
\addcontentsline{toc}{chapter}{目次}
\clearpage

\listoffigures
\addcontentsline{toc}{chapter}{図目次}
\clearpage

\listoftables
\addcontentsline{toc}{chapter}{表目次}
\clearpage

\printglossary
\addcontentsline{toc}{chapter}{記号一覧}
\clearpage

\pagebreak
\pagenumbering{arabic}
\pagestyle{fancy}
