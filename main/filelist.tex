\section{ファイル解説}
E-Sys用 \LaTeX2e 論文作成マクロ・ファイル集は,Table~\ref{table:filelist}に示すファイルから構成される.これらのマクロ・ファイルの適切な箇所を変更すると,論文の本文以外の部分が自動的に生成される.

\begin{table}[t]
\begin{center}
\caption{File and macro list for thesis making used in Our E-Sys labratory. Far right column shows the necessity for change of the file. A round mark (○) means that the file must be changed. A cross mark (×) means that the file must not be changed.}

%\vspace{3mm}
\begin{tabular}{l|l|c}\hline
%Filename & Explanation of file & Necessity for change \\\hline\hline
\multicolumn{1}{c|}{Filename} & \multicolumn{1}{c|}{Explanation of file} & \multicolumn{1}{c}{Necessity for change}\\ \hline\hline\hline
main.tex & The file which becomes main & ○ \\\hline
predoc.tex & A cover, acknowledgement, TOC, etc. & ×\\ 
 & ~~~are constituted &  \\\hline
esysthesis.sty & Style(Macro) file  & × \\\hline
esyshdr.sty & Style(Macro) file & × \\\hline
agumark.eps & The picture file of AGU mark & × \\\hline
esysthesis.bst & BibTeX Style File & × \\\hline
esysnomencl.ist & Symbol List Setting File & × \\\hline
esysnomencl.sty & Symbol List Style(Macro) File & × \\\hline\hline

masterzyuryouin.tex & The file for receipt stamp page file & ○ \\
 & ~~~of Master Thesis &  \\\hline
seniorzyuryouin.tex & The file for receipt stamp page file & ○\\
 & ~~~of Graduate Thesis &  \\\hline\hline

abstract.tex & Thesis summary is discrived in English & ○ \\\hline
achievement.tex & Achievements are discrived & ○ \\\hline
program.tex & Appendix: &  ○ \\
 & Programs and data are explained &  \\\hline
qa.tex & Appendix: & ○ \\
 & Questions and answers are described &  \\\hline
thanks.tex & Acknowledgement is filled in & ○ \\\hline
youshi.tex & Thesis summary is discrived in Japansese & ○ \\\hline\hline

Makefile & Procedure utilized by the make command & ○ \\
 & ~~~is described &  \\\hline\hline
 
eclbkbox.sty & This file may not exist & × \\\hline
fancyvrb.sty & This file may not exist & × \\\hline\hline
\end{tabular} 
\label{table:filelist}
\end{center}
\end{table}